% Назва дисертації

\title{Розроблення методів синтезу даних та генерації математичних комбінаторних задач за допомогою великих мовних моделей}
	
	% Прізвище, ім'я, по батькові здобувача
 
\author{Ніколаєв Андрій Дмитрович}
	
	% Факсимільний підпис автора у файлі sorokina-sig.pdf, .eps, .jpeg тощо
	% (зсув по x, зсув по y)
	% [параметри команди \includegraphics]
	% \facsimilesig{author}(-60,-12)[width=70pt]{sorokina-sig}
	
	% Прізвище, ім'я, по батькові наукового керівника/консультанта
 
\supervisor{Анісімов Анатолій Васильович}{доктор фізико-математичних наук, професор}
	
%	\speciality(uk){122}  % шифр спеціальності
%	\speciality(en){122}
	
	% Індекс за УДК
 
\udc{004.8}
	
	% Установа, де виконана робота, і місто
 
\institution{Київський національний університет імені Тараса Шевченка\\Міністерство освіти і науки України\\Київський національний університет імені Тараса Шевченка\\Міністерство освіти і науки України}{Київ}
	
	% Команду \council переписано в стилі команди \institution:
	% ключ institution задає «стандартну» назву установи, у спеціалізованій вченій раді якої проводиться захист дисертації
	% (з вказуванням назви органу, до сфери управління якого належить заклад, установа),
	% а ключ altname — «альтернативну» (тобто скорочену, для анотації).
	% Якщо факультативний аргумент відсутній, то клас вважає,
	% що захист проводиться в тій самій установі, де здійснювалася підготовка здобувача,
	% а отже, немає потреби писати назву цієї установи двічі на титульному і в анотації.
	% Але за потреби можна вручну повторити тут назву установи, і буде повтор на титульному і в анотації.
 
%	\council(uk){ДФ~41.052.021}
%	\council(en){ДФ~41.052.021}
	
	% Рік, коли написана дисертація
\date{2025}
	% LC_CTYPE=cp1251 luit grep -ro -P cite\{.*?\}
	
	% Тут буде титульний аркуш
\maketitle