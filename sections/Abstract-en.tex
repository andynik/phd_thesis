%
\addtocontents{toc}{\setcounter{tocdepth}{-1}}
\chapter*{Abstract}
\addtocontents{toc}{\setcounter{tocdepth}{2}}

\emph{Nikolaiev~A.D.} Development of Data Synthesis and Mathematical Combinatorial Problem Generation Methods Using Large Language Models. ~\textbf{\textemdash}~Qualification scientific work on the rights of the manuscript.

The PhD thesis on competition for a scientific degree of the doctor of philosophy in the speciality 122 \say{Computer~Science}.~\textbf{\textemdash}~Taras Shevchenko National University of Kyiv, Kyiv, 2025.

This dissertation investigates the use of large language models for data synthesis and the generation of mathematical combinatorial problems. The primary goal of the research is to investigate the mathematical reasoning capabilities of these models and to develop effective methods for generating synthetic data that preserves the mathematical essence of the problems.

The study introduces novel techniques for creating variations of mathematical combinatorial problems by modifying their configurations as well as their linguistic and stylistic characteristics. A series of experiments was conducted on the generated data to evaluate the impact of synthesised problems on the performance of large language models and to compare their outcomes with those achieved by experts with Olympiad-level mathematical experience.

\textbf{Main results and scientific contributions:}
\begin{itemize}
    \item A data synthesis method based on systematic manipulation of the texts of mathematical combinatorial problems was developed to compare the reasoning performance of large language models with that of experts possessing Olympiad-level experience.
    \item A novel method for generating variations of mathematical combinatorial problems was proposed. This approach involves the classification, selection, and creation of new synthetic variations that preserve the mathematical essence of the problems, utilising large language models and introducing a variation consistency metric.
\end{itemize}

The findings demonstrate the significant potential of large language models in generating combinatorial problems that maintain their mathematical integrity, thereby opening up new avenues for the automatic formalisation of mathematical texts. A key challenge remains to ensure the accuracy of generated solutions; as shown in the experimental section, language models exhibit high sensitivity to modifications in the problem texts, including the addition of irrelevant numerical information, changes in problem configuration, and linguistic-stylistic alterations. To further improve automated proof-search systems, the dissertation proposes integrating language models with formal symbolic computation techniques.

Based on the experimental studies, the following achievements were realised:
\begin{enumerate}
    \item A comprehensive review of artificial intelligence systems and modern natural language processing methods was conducted. This included an analysis of various model architectures, reasoning techniques, the use of auxiliary tools for symbolic data processing, as well as existing datasets and evaluation metrics.
    \item The \emph{Combi-Puzzles} dataset was developed, comprising 125 combinatorial problems with systematic modifications of problem conditions by controlling parameters such as problem configuration, the injection of redundant information, and changes in the linguistic-stylistic format of the problem statements.
    \item An experimental comparison of the model performance in solving mathematical combinatorial problems on synthesised data was performed. Approximately 36,000 model responses were evaluated using a set of criteria designed to verify the correctness of logical reasoning during the generation of problem statements, along with an assessment of the models' sensitivity to text modifications.
    \item A series of experiments involving 35 participants with Olympiad-level experience was carried out, and an analysis of over 800 problem solutions that were conducted for the comparative evaluation of the performance of models and experts.
    \item Utilising the developed methods for the selection, generation, and quality evaluation of synthetic data for combinatorial problems, more than 20,000 instances of mathematical combinatorial problems were generated.
\end{enumerate}

\emph{Keywords}: artificial intelligence, automated theorem proving, natural language processing, large language models, machine learning, mathematical problems.
