\chapter*{Висновки}

У даній дисертаційній роботі проведено комплексне дослідження можливостей великих мовних моделей для синтезу даних, генерації математичних комбінаторних задач та автоматичної формалізації математичних текстів. Результати роботи підтверджують актуальність і доцільність застосування сучасних нейромережевих підходів для вирішення завдань математичного міркування. Основні наукові досягнення роботи можна підсумувати наступним чином:
\begin{itemize}
    \item Проведено огляд стану сучасних методів для обробки природної мови та моделей штучного інтелекту для роботи з математичними задачами, зокрема архітектур моделей для генерації текстів, методів машинного навчання, інженерії запитів при роботі з мовними моделями, задіяння додаткових інструментів для символьної обробки даних, існуючих наборів даних та метрик оцінювання. Виділено особливості різних типів математичних задач та проведено повіряння ефективності моделей у генерації розв'язків за допомогою версій мовних моделей GPT-4, Qwen, LLaMA та Mistral.
    \item Розроблено набір даних \emph{Combi-Puzzles}, який включає набір з комбінаторних задач з систематичною модифікацією умов за рахунок керування наступними параметрами та особливостями задач: конфігурація параметрів задачі, додавання зайвої інформації, зміна лінгвістично-стилістичної формату тексту умов задач.
    \item Проведено експериментальне порівняння ефективності моделей до розв'язання математичних комбінаторних задач на синтезованих даних та оцінено більш ніж 36 тис. із залученням сучасних ВММ із різними архітектурами та інтегрованими методами для проведення математичного міркування, зокрема з версіями моделей GPT-4, LLaMA, Mixtral у кількох розмірах та специфікаціях.
    \item Проведено експериментальну частину з залученням людей-експертів, які мають олімпіадний математичний досвід. Результати експертів додані як метрика для оцінювання ефективності роботи ВММ. Порівняльний аналіз показав, що мовні моделі здатні демонструвати високу точність у розв'язанні задач заданих на вхід у формальному вигляді, водночас з цим їхня ефективність суттєво знижується за наявності додаткового інформаційного шуму та стилістичних змін текстів задач.
    \item Запропоновано новий метод для генерації синтетичних варіацій математичних комбінаторних задач через систематичну маніпуляцію текстів за рахунок контролю параметрів задач. Для оцінювання якості згенерованих даних запроваджено нову метрику \emph{Показник варіаційної узгодженості}, яка дозволяє оцінити якість генерованих синтетичних даних через аналіз частоти правильних розв'язків і коефіцієнтів кореляції між варіаціями задач. Цей підхід сприяє розширенню навчальних корпусів і підвищенню варіативності даних.
\end{itemize}

Отримані результати підтверджують виконання поставлених завдань і досягнення загальної мети дослідження щодо синтезу даних та генерації математичних комбінаторних задач за допомогою великих мовних моделей.

Проведене дослідження демонструє високий потенціал великих мовних моделей у генерації математичних текстів і вдосконаленні систем автоматизованого пошуку доведень. Розроблені методи не лише підвищують ефективність методів генерації синтетичних математичних задач, а й відкривають нові можливості для інтеграції нейромережевих підходів із формальними методами. Отримані результати становлять важливий внесок у сучасний напрям дослідження штучного інтелекту, який спрямований на розвиток методів обробки і генерації даних, а також є важливим для подальшої інтеграції відповідних систем у навчальні середовища, тести та платформи для дистанційного навчання, що сприяє її використанню у реальних освітніх процесах.

Таким чином, дисертаційна робота підтверджує досягнення поставленої мети та сприяє подоланню існуючих викликів у цій галузі. Отримані результати закладають основу для подальших наукових розробок у сфері обробки природної мови та штучного інтелекту, спрямованих на ефективну роботу з математичними даними.
