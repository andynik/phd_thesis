\chapter*{Вступ}

\paragraph{Обґрунтування вибору теми дослідження.}

Математичне міркування є фундаментальним аспектом освіти та наукових досліджень. Здатність розв'язувати складні математичні задачі є критичною не лише в математиці, але й у різних галузях, таких як фізика, інженерія, комп'ютерні науки, економіка, біологія тощо. Традиційно, математичне міркування вимагало людської експертизи, значного обсягу досвіду та вміння працювати з формальними типами даних.

З розвитком штучного інтелекту (ШІ) та методів обробки природної мови (Natural Language Processing, NLP) з’явилися великі мовні моделі (ВММ), які демонструють вражаючі результати в аналізі, розумінні та генерації тексту. Сучасні моделі, такі як новітні версії GPT і LLaMA, досягли значного прогресу у вирішенні різноманітних завдань у роботі з текстом, зокрема й з математичними задачами.

Застосування ВММ для синтезу та генерації математичних текстів відкриває нові можливості. З одного боку, ВММ можуть обробляти великі обсяги даних та виявляти закономірності у великих масивах даних та приходити до висновків, які можуть використовувати задля автоматичної формалізації текстів та генерації розв'язань задач. Проте з іншого боку, математичне міркування вимагає точних математичних обчислень, дедукції, а також можливості працювати з абстрактними концепціями. Незважаючи на те, що великі мовні моделі демонструють вражаючі результати у розв'язанні математичних задач, більшість цих результатів базується на запропонованих наборах задач та не може бути ефективно доводити адекватність моделей. Для підтвердження даного факту необхідно створити унікальний набір математичних задач, який не було використано для попереднього навчання даних моделей.

Задля ефективної демонстрації обмеження моделей та оцінки їхніх можливостей, у даній роботі було проведено дослідження з математичними комбінаторними задачами, які незважаючи на звичайний текстовий вигляд мають прихований математичний підтекст -- питання або задачу, що потребує здатності моделей до ефективного міркування та пошуку відповідних підходів для розв'язання.

Виконання поставленої задачі передбачає огляд сучасного стану проблеми, підготовки даних, проведення експериментів з залученням людини та сучасних версій ВММ з проведенням аналітичної частини дослідження.

\paragraph{Мета та завдання дослідження}
Метою дисертаційної роботи є дослідити застосування великих мовних моделей для розробки методів синтезу математичних даних та генерації математичних комбінаторних задач.
Для досягнення поставленої мети було поставлено наступні завдання:

\begin{enumerate}
    \item Проаналізувати сучасний стан моделей штучного інтелекту та методів обробки природної мови для роботи з математичними задачами, що включає огляд архітектур, формальних методів, методів машинного навчання, існуючих наборів даних та метрик для оцінювання ефективності роботи моделей.
    \item Розробити власний набір математичних комбінаторних задач з новими варіаціями через систематичні модифікації текстів оригінальних задач.
    \item Провести експериментальне дослідження для визначення впливу характеристик математичних задач на ефективність роботи великих мовних моделей із залученням експертної оцінки.
    \item Розробити та впровадити метод генерації синтетичних текстів задач за допомогою великих мовних моделей.
    \item Здійснити порівняльний аналіз ефективності моделей у генерації математичних текстів.
\end{enumerate}

\textbf{Об'єктом дослідження} є моделі штучного інтелекту та методи обробки природної мови для роботи з математичними текстовими задачами.
\medskip

\textbf{Предметом дослідження} є методи синтезу даних та генерації математичних комбінаторних задач за допомогою великих мовних моделей.
\medskip

\textbf{Методи дослідження} включать: методи обробки природної мови з використанням багатошарової нейромережевої архітектури на основі людського зворотного зв’язку, навчання з підкріпленням, моделей-експертів та інших підходів; техніки побудови запитів; методи статичного аналізу та обробки даних.

Обчислювальні ресурси та інструменти: Експерименти проводилися з використанням графічних процесорів Nvidia A100, Nvidia Quadro RTX 8000, а також серверних потужностей компанії OpenAI з доступом до моделей через API; для експериментів використовувалися версії великих мовних моделей GPT-4, LLaMA, Qwen, Mistral у кількох версіях та розмірах; для розробки програмного забезпечення було використано мову програмування Python 3.11 із відповідними бібліотеками для роботи з даними та мовними моделями.

\medskip

\paragraph{Наукова новизна отриманих результатів}

У дисертаційній роботі проведено комплексне дослідження можливостей великих мовних моделей для генерації математичних комбінаторних задач, зокрема порівняння з ефективності різноманітних мовних моделей, людей-експертів, проведення аналізу числових та стилістичних особливостей текстів задач. Запропоновано нові методи генерації синтетичних даних та метрики оцінки якості згенерованих даних.

Протягом вирішення поставлених задач дослідження, вперше отримані наступні результати:
\begin{itemize}
    \item Розроблено метод синтезу математичних комбінаторних задач зі збереженою сутністю шляхом відбору та систематичної маніпуляції текстів задач за рахунок контролю параметрів та стилістичних особливостей задач, на основі якого створено новий набір задач.
    \item Вперше проведено експеримент з участю людей-експертів з олімпіадним математичним досвідом задля проведення оцінювання здатності моделей до розв'язання варіацій комбінаторних задач та чутливості до модифікацій текстів.
    \item Запроваджено нову метрику варіаційної узгодженості текстів задач, що дозволяє оцінювати якість синтетичних даних та їхню відповідність оригінальним текстам.
\end{itemize}

\paragraph{Важливість та практичне значення отриманих результатів}

Отримані результати підтверджують високий потенціал великих мовних моделей у генерації математичних текстів та удосконаленні систем автоматизованого пошуку доведень. Розроблені методи синтезу даних та генерації математичних комбінаторних задач не лише підвищують ефективність автоматизованих підходів до створення математичних задач, а й відкривають нові можливості для інтеграції нейромережевих технологій із формальними методами обчислень.

Проведена робота є важливим внеском у сучасні дослідження штучного інтелекту, які спрямовані на автоматизацію математичної освіти та розвиток ефективних систем доведень, і створюють основу для подальших наукових розробок у сфері обробки природної мови та аналізу математичних даних.

Окрім використання отриманих результатів для подальшого поліпшення математичних можливостей ВММ, вони закладають основу до розробки адаптивних навчальних систем та інтеграції моделей штучного інтелекту у освіту та наукові дослідження.

% Перевірити та оновити перед публікацією.
\paragraph{Структура та обсяг дисертаційної роботи}

Дисертація складається зі вступу, чотирьох розділів, кожен з яких завершується підсумками, висновків, списку використаних джерел та додатків. Загальний обсяг роботи складається з 142 сторінок, з яких основний зміст викладено на 112 сторінках. Робота містить 44 рисунка, 25 таблиць та список використаних джерел посилань із 91 найменувань.

\newpage